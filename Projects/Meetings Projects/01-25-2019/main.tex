\documentclass[12pt]{article}

\usepackage{physics}
\usepackage{siunitx} 
\usepackage{enumerate} 
\usepackage{hyperref}
\usepackage{color}
\usepackage{listings}
\usepackage{enumitem,amssymb}
\newlist{todolist}{itemize}{2}
\setlist[todolist]{label=$\square$}
\definecolor{lightgray}{rgb}{.9,.9,.9}
\definecolor{darkgray}{rgb}{.4,.4,.4}
\definecolor{purple}{rgb}{0.65, 0.12, 0.82}

\lstdefinelanguage{JavaScript}{
  keywords={typeof, new, true, false, catch, function, return, null, catch, switch, var, if, in, while, do, else, case, break},
  keywordstyle=\color{blue}\bfseries,
  ndkeywords={class, export, boolean, throw, implements, import, this},
  ndkeywordstyle=\color{darkgray}\bfseries,
  identifierstyle=\color{black},
  sensitive=false,
  comment=[l]{//},
  morecomment=[s]{/*}{*/},
  commentstyle=\color{purple}\ttfamily,
  stringstyle=\color{red}\ttfamily,
  morestring=[b]',
  morestring=[b]"
}

\lstset{
   language=JavaScript,
   backgroundcolor=\color{lightgray},
   extendedchars=true,
   basicstyle=\footnotesize\ttfamily,
   showstringspaces=false,
   showspaces=false,
   numbers=left,
   numberstyle=\footnotesize,
   numbersep=9pt,
   tabsize=2,
   breaklines=true,
   showtabs=false,
   captionpos=b
}


\begin{document}

\title{ Setting Up Node.js On Ubuntu VM}
\author{Ramiro Gonzalez}
\date{January 24, 2019}

\maketitle

\begin{Objective}
	This project will consist of working with the Linux Command Line and VIM (text editor). It is assumed that you are running a Ubuntu Distro (Linux Distribution), have set up LAMP. The end goal of this project is to successfully install Node.js and be able to set up a server. Learn to use VIM. 
\end{Objective}

\section*{Background}
	As the name implies Node.js is a JavaScript runtime environment, that is it enables us to create dynamic web page content before sending it to the users browser. Node.js is not a programming language it is a JavaScript runtime environment. The Linux Command Line is fast an efficient and should be used for all tasks.  
\section*{Material}
 \color{black}
\begin{todolist}
    \item Computer 
    \item Internet Connection
    \item Ubuntu Linux (Linux Distribution) 
\end{todolist}
\section{Tasks}
Complete the following tasks. Note that different operating systems may require different steps. 
\section*{Task 1}
First create a folder named playground. This folder is where you will store your exercise files. Make sure you know how to move around. \\
\begin{todolist}
	  \item Open a terminal
	  Navigate around the terminal using the following commands. 
	  \begin{itemize}
	       \item pwd (Print working directory.Current directory) 
	      \item cd (Change directory) 
	      \item cd $~$(Change directory to home)
	      \item mkdir playground (Create a directory [folder] named playground) 
	      \item rm -r playground (Delete the directory [folder] named playground)
	  \end{itemize}
	  \item Create a directory (folder) named playground
	  \begin{lstlisting}
	  $ mkdir playground
	  \end{lstlisting}
	  \item Make the current directory playground 
	  \begin{lstlisting}
	  $ cd playground
	  \end{lstlisting}
\end{todolist}
\section*{Task 2} 
Installing node.js is as follows.  The version may be different. We assume your repositories are working. 
\begin{todolist}
    \item Type the following 
    \begin{lstlisting}
    $ sudo apt install nodejs
    \end{lstlisting}
    \item Check the version. 
    \begin{lstlisting}
    $ nodejs -v
    \end{lstlisting}
    \item If the above did not work consider the following. \\
    \href{https://github.com/nodesource/distributions/blob/master/README.md}{https://github.com/nodesource/distributions/blob/master/README.md}
\end{todolist}
\section*{Task 3}
We will now write a simple javaScript program and run it using node. 
\begin{todolist}
    \item Create a JavaScript program that ouputs "Hello, Universe". vim is a text editor.  
    \begin{lstlisting}
    $ vim hello.js 
    \end{lstlisting} 
    \item Once the program has been created run as follows. 
    \begin{lstlisting}
    $ node hello.js
    \end{lstlisting}
\end{todolist}
\section*{Task 4}
 First find your IP address. If you are using a Ubuntu Server with no Graphical User Interface you must configure it. 
\begin{todolist}
    \item (Using Ubuntu Server) Open VirtualBox and Click on your Server 
    \item (Using Ubuntu Server) Machine $>$ Settings $>$ Network $>$ Adapter 1 > Bridge Adapter
    \item On terminal Run the following command. 
    \begin{lstlisting}
    $ ifconfig 
    \end{lstlisting}
    \item If ifconfig is not installed do the following. 
    \begin{lstlisting}
    $ sudo apt install net-tools
    \end{lstlisting}
    \item Find your local IP address. If you are connected to UCM network. It should start with 10, Take note of it. 
    \begin{lstlisting}
    inet 10.x.x.x
    \end{lstlisting} 
    \item Take note of this IP address. 
    \item 127.0.0.0 is called the localhost 
    
\end{todolist}

\section*{Task 5}
Now we will set up a server. If you have a firewall installed you must open the necessary port. 
\begin{todolist}
    \item Create a file named server.js 
    \begin{lstlisting}
    $ vim server.js 
    \end{lstlisting} 
    Inside server.js file 
    \begin{lstlisting}
    var http = require("http");
    http.createServer(function(request,response){
        response.writeHead(200,{"Content-Type":"text/plain"});
        response.write("Hello, Universe");
        response.end();
    }).listen(8888);
    \end{lstlisting} 
\item Run the server using node. To exit press CTRL-C
\begin{lstlisting}
$ node server.js 
\end{lstlisting}
\item Open A browser and type 127.0.0.0:8888 or \href{http://localhost:8888}{http://localhost/}
\item (Ubuntu Server) Find a computer or use your browser. Use your local IP address. 
\begin{lstlisting}
10.x.x.x:8888
\end{lstlisting}
\end{todolist}
\section*{Summary}
You have set up the necessary components for working with JavaScript. By now you should be able to move around the Linux File System, create files using VIM text editor, and able to create a server as well as access it through the browser. It is important that you take note of your local IP address, this IP is assigned by the network you are connected to. 
\end{document}